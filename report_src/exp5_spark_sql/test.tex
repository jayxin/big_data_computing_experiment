%%%%%%%%%%%%%%%%%%%%%%%%%%%%% Define Article %%%%%%%%%%%%%%%%%%%%%%%%%%%%%%%%%%
\documentclass{article}
%%%%%%%%%%%%%%%%%%%%%%%%%%%%%%%%%%%%%%%%%%%%%%%%%%%%%%%%%%%%%%%%%%%%%%%%%%%%%%%

%%%%%%%%%%%%%%%%%%%%%%%%%%%%% Using Packages %%%%%%%%%%%%%%%%%%%%%%%%%%%%%%%%%%
\usepackage{ctex}
\usepackage{geometry}
\usepackage{graphicx}
\usepackage{pgfplots}
\usepackage{float}
\usepackage{minted}
\usepackage{hyperref}
\hypersetup{
  colorlinks=true,
  pdfstartview=Fit,
  pdfcreator={Shit},
pdfproducer={Big shit}}
%\usepackage{amssymb}
%\usepackage{amsmath}
%\usepackage{amsthm}
%\usepackage{empheq}
%\usepackage{mdframed}
%\usepackage{booktabs}
%\usepackage{lipsum}
%\usepackage{color}
%\usepackage{psfrag}
%\usepackage{bm}
%%%%%%%%%%%%%%%%%%%%%%%%%%%%%%%%%%%%%%%%%%%%%%%%%%%%%%%%%%%%%%%%%%%%%%%%%%%%%%%

% Other Settings

%%%%%%%%%%%%%%%%%%%%%%%%%% Page Setting %%%%%%%%%%%%%%%%%%%%%%%%%%%%%%%%%%%%%%%
\geometry{a4paper}

%%%%%%%%%%%%%%%%%%%%%%%%%%%%%%% Plotting Settings %%%%%%%%%%%%%%%%%%%%%%%%%%%%%
\usepgfplotslibrary{colorbrewer}
\pgfplotsset{width=8cm,compat=1.18}
%%%%%%%%%%%%%%%%%%%%%%%%%%%%%%%%%%%%%%%%%%%%%%%%%%%%%%%%%%%%%%%%%%%%%%%%%%%%%%%

%%%%%%%%%%%%%%%%%%%%%%%%%%%%%%% Title & Author %%%%%%%%%%%%%%%%%%%%%%%%%%%%%%%%
\title{实验五: SPARK SQL 基础编程方法二}
\author{胡嘉鑫 \and 102102145}
\date{\today}
%%%%%%%%%%%%%%%%%%%%%%%%%%%%%%%%%%%%%%%%%%%%%%%%%%%%%%%%%%%%%%%%%%%%%%%%%%%%%%%

\begin{document}
\maketitle
\tableofcontents

\section{实验目的}
\begin{itemize}
  \item 理解 SPARK 工作流程;
  \item 掌握 SPARK SQL 基础编程方法;
\end{itemize}

\section{实验平台}
\begin{itemize}
  \item OS: Linux
  \item Hadoop v3.1.3
  \item JDK v1.8
  \item Spark v3.4.0
\end{itemize}

\section{实验步骤}
\subsection{用户搜索前三名统计}
\subsubsection{Problem Description}
用户搜索日志记录文件为 UserLogHot.txt, 格式为日期, 用户 id, 商品, 用户区域,
搜索终端;
\begin{enumerate}
  \item 查询每天每用户点击某商品的次数, 输出到屏幕.
  \item 将用户每天点击商品的次数汇总累加,
    统计每用户在每天点击搜索每个商品的总次数,
    输出 JSON 格式文件, JSON 元数据格式如下:
    Date(日期)、UserID(用户 id)、Item(商品)、Count(该用户在当天点击该商品的次数)
  \item 统计出用户搜索每商品的前 $ 3 $ 名:
    用步骤 $ 2 $ 的 JSON 字符串, 构造DataFrame。
    在 Spark SQL 注册临时表, 使用窗口函数 row\_number 统计出每用户搜索每商品的前
    $ 3 $ 名, 将结果以 JSON 格式输出到屏幕或者文件。
\end{enumerate}

\subsubsection{Code}
\subsubsection{Result}

\subsection{NBA 数据统计}
\subsubsection{Problem Description}
\begin{enumerate}
  \item 根据胜负场次统计 $ 15 $ - $ 16 $ 赛季都能够进入季后赛的球队
    (查询东西部联盟排名前 $ 8 $ 的球队)
  \item 针对最近两个赛季常规赛排名, 根据排名的变化, 找出东、西部进步幅度最大的球队。
  \item 分析统计出 $ 15 $ - $ 16 $ 赛季命中率(第三列)、 三分球命中率(第六列)、
    防守能力最强(失分越少防守越强)的三支球队.
\end{enumerate}

\subsubsection{Code}
\subsubsection{Result}

\section{出现的问题及其解决方案}
没有问题.
\end{document}
